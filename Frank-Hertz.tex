\documentclass[11pt]{article}
\usepackage{graphicx}\graphicspath{ {./images/} }
\usepackage{mathtools,amssymb,amsthm}
\usepackage{tabularx}
\usepackage{hyperref}
\usepackage{pgfplots}
\pgfplotsset{width=14cm,compat=1.9}

\hypersetup{
    colorlinks=true,
    linkcolor=blue     
    }
    
\begin{document}

\begin{titlepage}
    \begin{center}
        \vspace*{1cm}
            
        \Huge
        \textbf{Franck-Hertz Mercury and Neon Experiment}
                       
        \vspace{1.5cm}
    
\includegraphics[scale=0.7]{Screenshot 2023-12-12 212432.png}
\includegraphics[scale=0.8]{Screenshot 2023-12-12 212452.png}
\includegraphics[scale=0.6]{Screenshot 2023-12-12 212501.png}
\\
        \vspace{0.5cm}    
        \large
        \textbf{Name: SM Shouvik Raj\\
        Lab partner: Osebi Daudu\\
        NSID: dew873\\
        Student Number: 11339451}\\            
    \end{center}
\end{titlepage}

\newpage
\section{Mercury Franck-Hertz Experiment}
\subsection{Data}
\begin{align*}
\begin{tabularx}{0.5\textwidth} { 
  | >{\centering\arraybackslash}X 
  | >{\centering\arraybackslash}X | }
 \hline
$V_{a}(V)$ & $I(A)$\\
 \hline
 1.5 & 0.1\\
 3.0 & 0.2\\
 4.5 & 0.3\\
 6.0 & 1.0\\
 7.5 & 1.1\\
 7.8 & 1.2\\
 9.0 & 0.9\\
 10.5 & 1.9\\
 11.4 & 2.4\\
 12.0 & 2.2\\
 13.5 & 1.2\\
 15.0 & 2.1\\
 16.2 & 3.5\\
 16.5 & 3.7\\
 18.0 & 2.2\\
 19.5 & 2.1\\
 21.0 & 4.9\\
 21.9 & 5.5\\
 22.5 & 5.0\\
 24.0 & 3.9\\
 25.0 & 6.3\\
\hline
\end{tabularx}
\\
\end{align*}
\\
\subsection{Mercury Graph}
\begin{tikzpicture}
    \begin{axis}[
    title={Graph of collected current $I(A)$ against accelerating potential $V_a$(V)},
    xlabel={$V_a$(V)},
    ylabel={$I(A)$},
    xmin=0, xmax=30,
    ymin=0, ymax=7,
    xtick={0,5,10,15,20,25,30},
    ystick={0,1,2,3,4,5,6,7},
    ymajorgrids=true,
    xmajorgrids=true,
    grid style=lined
    ]
    \addplot[
    color=blue,
    mark=square
    ]
    coordinates {(1.5,0.1)(3,0.2)(4.5,0.3)(6,1)(7.5,1.1)(7.8,1.2)(9,0.9)(10.5,1.9)(11.4,2.4)(12,2.2)(13.5,1.2)(15,2.1)(16.2,3.5)(16.5,3.7)(18,2.2)(19.5,2.1)(21,4.9)(21.9,5.5)(22.5,5)(24,3.9)(25,6.3)
    };
\end{axis}
\end{tikzpicture}
\subsection{Analysis}
\\
Average Peak separation $\Delta V=\frac{11.8-7.4+16.5-11.4+21.9-16.5}{3}=4.8333\;V$
\begin{align*}
    Using\;\;\;\;\Delta E&=eV\;\;\;where\;\;e=1.602\times 10^{-19}C\\
    &=1.602\times 10^{-19}\times 4.8333\\
    &=7.743\times 10^{-19}\\
    Then\;\;\;\lambda&=\frac{hc}{\Delta E}\\
    &=\frac{6.626\times 10^{-34}\times2.998\times10^8}{7.743\times10^{-19}}\\
    &=257\times10^{-9}m\\
    &\approx257\;nm\\ Percentage\;error\;with\;&accepted\;value\;of\;253.7\;nm:\\
    &\Bigg|\frac{253.7-257}{253.7}\Bigg|\times100\%\\
    &=1.30\%
\end{align*}
\\
\subsection{Discussion}
The percentage error calculated above is due to the inconsistent nature of increasing accelerating potential manually. ALso resistance within the electrical circuit (which potentially varies non-linearly with voltage) could  also be at play since if a larger peak separation value was obtained that would increase $\Delta E$ thus returning a smaller value for $\lambda$. \\
The first excitation peak is noticed at about 7.4V while the average peak separation is 4.83V. This is due to the contact potential between the cathode and anode within the tube.\\
The current drops gradually at the critical energies due to electrons making it to the collector electrode going through the mercury vapor without interacting with the atoms. I would expect to see similar results for the second and higher excited states since this experiment shows electrons need a specific amount of energy for state transition and the number of orbitals traversed is directly related to this amount of energy.  

\newpage
\section{Neon Franck-Hertz Experiment}
\subsection{Data}
\textit{Data included separately due to being too long.}
\subsection{Graph}
\includegraphics[scale=0.6]{Screenshot 2023-12-12 191412.png}
\subsection{Analysis}
Average Peak separation from graph:
\begin{align*}
    \DeltaV&=\frac{55.4-19.15}{2}\;\;\;(common\; terms\; cancel\; out)\\
    &=18.125\;V\\
    Then\;using\;\;\;\Delta E&=eV\\
    &=18.125e\\&=2.904\times 10^{-18}\;J\\
    Finally\;\;\;\lambda&=\frac{hc}{\Delta E}=\frac{hc}{2.904\times 10^{-18}}\\
    &=6.8\times10^8\;m\\
    &\approx68.0\;nm\\
    For\;expected\;neon\;bands\;at\;16.7eV;\\
    \lambda&=\frac{hc}{16.7eV}\\
    &\approx74.0\;nm\\
    Percentage\;error&:\;\;
    \Bigg|\frac{74-68}{74}\Bigg|\times100\%\\
    &=8.11\%
\end{align*}
\subsection{Discussion}
The glow seen in the neon tube is orange-red and it first appears around the centre and moves towards the cathode as potential increases. The previous experiment with mercury also emits light but in the ultraviolet range (257nm from experiment). More bands of orange-red light form as the potential goes through successive values spaced roughly 18.125V apart as obtained from the experiment.\\
The wavelength calculated for the expected orange-red glow comes out to be 68.0 nm. Which is closer to the wavelength UV and smaller by about a magnitude of 10 compared to the actual wavelength of light of this color. Furthermore the the band of neon states seen for 16.7eV results in light of wavelength 74.0 nm, which is also 10 times smaller than actual wavelength of the light observed. The reason for this phenomenon is partially explained in the footnote of this \href{https://books.google.ca/books?id=xQfKWwvH42kC&pg=PA32&redir_esc=y#v=onepage&q&f=false}{page}.
\subsection{Conclusion}
To conclude the results of this experiment clearly show the quantum nature of atoms by displaying how excitations of electrons happen at constant potential intervals and that the energy transfer due to the size of this potential is equal to the wavelength emitted by the target element. 
\end{document}
