\documentclass[12pt, letterpaper]{report}
\usepackage{graphicx}\graphicspath{ {./images/} }
\usepackage{mathtools,amssymb,amsthm}
\usepackage{tabularx}
\usepackage{multirow}
\title{Millikan Oil Drop}
\author{SM Shouvik Raj
\\NSID: dew873
\\Student Number: 11339451
\\ Lab Partner: Osebi Daudu}
\date{November 4th 2023}
\begin{document}
\maketitle
\noindent
\textbf{Objective:}
\\\\
This experiment is designed to show the quantization of electric charge and allow determination of the
elementary charge, \textit{e}.
Oil drops are sprayed into a region where a uniform electric field is
established and drops that are held motionless in the field are analysed.
It is only through an
analysis of the data that the elementary charge can be determined. A number of drops should be observed
and their respective charges calculated. If the charges on these drops are integral multiples of a certain
smallest charge, then this is a good indication of the quantum nature of electricity. 
\\\\
\textbf{Apparatus:}
\\
\includegraphics[]{Screenshot 2023-10-30 175325.png}
\\\\
The equipment consists of a stand supporting the oil drop chamber, the light, and
the measuring microscope; and a power supply for the light and the voltage to produce the electric field in
the chamber. In addition, an electronic digital stopwatch is provided for measuring the rise and fall times
of the oil drops.
An atomizer is used to produce the oil drops. The nozzle of the atomizer is placed against the two
boreholes of the chamber. A quick squirt will fill the chamber with drops which become visible in the
viewing area.
\\\\
\textbf{Procedure:}\\
The simulation is located at https://www.busybwebdesign.com/lab$_$simulati\\ons/Millikan/
Read the instructions on the first page, then click Begin.
To spray oil drops into the chamber, click on the spritzer bulb. If an oil drop remains suspended between
the plates, click the scope to view the drop. If no oil drop remains suspended, click the bulb again.
Measure the radius of the drop (and record the uncertainty) and record the oil density, plate separation,
and the voltage between the plates. Record the uncertainty in the voltage.
Collect radius and voltage data for a total of 10 oil drops.
Calculate the charge on each of your drops.
To attempt to show quantization of charge, average your smallest drop charges, divide this charge value
into your other drop charges, and see if you obtain integer factors.
Based on your data and analysis, calculate the value of the elementary charge and compare to the accepted
value of $1.602\times10^{–19} C$.
\\
\newpage
\noindent \textbf{Simulation Data and Analysis} (since limited successful physical ex data available)
\\\\
Since,\\
\begin{equation*}
    mg=qE
\end{equation*}
\\
Solving for $q$ with $m=\frac{4}{3}\pi r^3 \rho \;and\;E=\frac{\Delta V}{d}$:
\\
\begin{equation*}
    q=\frac{mg}{E}=\frac{4\pi r^3 \rho g d}{3\Delta V}
\end{equation*}
\\ 
Where $r = radius\;of\;the\; drop$, $\rho = oil\;density$ and $d = capacitor\;plate\;separation$
\\\\
Calculation for $q$ for first drop:
\\
$Oil\; Density:\;900kgm^{-3}$\\
$Capacitor\;plate\;separation:\;3mm$\\
\begin{align*}
    q&=\frac{4\pi (470\times 10^{-9})^3 \times 900 \times 9.81 \times 3\times 10^{-3}}{3\times 10.3}\\
    &=1.118\times 10^{-18}
\end{align*}
\\
\begin{align*}
\begin{tabularx}{0.8\textwidth} { 
  | >{\centering\arraybackslash}X 
  | >{\centering\arraybackslash}X 
  | >{\centering\arraybackslash}X
  | >{\centering\arraybackslash}X | }
 \hline
  & radius (nm) \pm\;5 & \Delta V (V) \pm\;0.01& q (C) \\
 \hline
 1  & 470& 10.3& $1.118\times 10^{-18}$\\
 2  & 340& 14.4& $3.028\times 10^{-19}$\\
 3  & 500& 10.7& $1.296\times 10^{-18}$\\
 4  & 490& 11.8& $1.106\times 10^{-18}$\\
 5  & 360& 16.8& $3.081\times 10^{-19}$\\
 6  & 290& 17.0& $1.592\times 10^{-19}$\\
 7  & 420& 16.9& $4.864\times 10^{-19}$\\
 8  & 380& 12.5& $4.870\times 10^{-19}$\\
 9  & 490& 16.7& $7.816\times 10^{-19}$\\
 10 & 550& 16.7& $1.10\times 10^{-18}$\\
\hline
\end{tabularx}
\\
\end{align*}
\\\\
Next, using the sixth charge value from the table and dividing all the other charges to check for integer values:
\begin{equation*}
    \frac{q_7}{q_6}=\frac{4.864\times 10^{-19}}{1.592\times 10^{-19}}=3.055 
\end{equation*}
\begin{align*}
\begin{tabularx}{0.4\textwidth} { 
  | >{\centering\arraybackslash}X 
  | >{\centering\arraybackslash}X | }
 \hline
    No. & charge ratio (N)\\
 \hline
 1  & 7.02\\
 2  & 1.90\\
 3  & 8.14\\
 4  & 6.95\\
 5  & 1.94\\
 6  & 1.00\\
 7  & 3.06\\
 8  & 3.06\\
 9  & 4.91\\
 10 & 6.94\\
\hline
\end{tabularx}
\end{align*}
\\
The ratios result in near integer value as seen from the table. Most likely due to uncertainty present in eyeballing the radius of the drops on a scale with 10 units per division and uncertainty of $\pm 5$.
\\\\\\
\textit{Elementary charge calculation:}\\\\
Using $e=\frac{q}{N}$ for first charge;\\
\begin{align*}
e_1=\frac{q_1}{N_1}=\frac{1.118\times 10^{-18}}{7}=1.597\times 10^{-19}C
\end{align*}
\\
Average: 
\begin{align*}
e=&\frac{(1.59+1.514+1.620+1.58+1.540+1.592+1.621+1.623+1.563+1.571)\times 10^{-19}}{10}\\=&1.5821\times 10^{-19} C
\end{align*}
\\
Percentage error with accepted elementary charge:\\\\
$\Big|\frac{1.602\times 10^{-19}-1.582\times 10^{-19}}{1.602\times 10^{-19}}\Big|\times 100\%=1.24\%$
\\\\
\textbf{Discussion:}\\
Taking measurement uncertainty into account and given oil density of $900kgm^{-3}$, when density of the oil can range from 700 to 950 $kgm^{-3}$, the obtained charge ratios can be used to prove quantization of charge.\\
Obtained elementary charge of $1.5821\times 10^{-19} C$ is smaller than the accepted value of $1.602\times 10^{-19}C$ with a percentage error of $1.24\%$ which is well within acceptable error range. \\\\
To eliminate as much uncertainty as possible, more accurate values for oil density based on atmospheric pressure should be taken as well as the simulation should allow for zoom function to obtain precise measurement data from the onscreen scope.
\\\\\\
\textbf{Conclusion:}\\
The results of this experiment is evidence for the existence of an elementary charge and thus quantized charges paving the way for quantum mechanical ideas of wave-particle duality. The physical experiment also enabled determination of the electron mass.        
\newpage
\noindent \textbf{Physical Experiment Data:}\\\\
\includegraphics[scale=0.23]{IMG_0595.jpg}
\end{document}