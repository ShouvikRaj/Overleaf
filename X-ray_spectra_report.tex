\documentclass{article}
\usepackage{graphicx}\graphicspath{ {./images/} }
\usepackage{mathtools,amssymb,amsthm}
\usepackage{tabularx}

\title{X-ray Spectra}
\author{SM Shouvik Raj
\\NSID: dew873
\\Student Number: 11339451
\\ Lab Partner: Osebi Daudu}\\
\begin{document}
\maketitle
\newpage
\section{Molybdenum X-ray spectrum}
\subsection{Schematic Apparatus}
\includegraphics[]{Screenshot 2023-12-11 193320.png}
\subsection{Data}
\includegraphics[]{Screenshot 2023-12-11 190736.png}
\subsection{Graphs}
\includegraphics[scale=0.5]{Screenshot 2023-12-11 191559.png}\\
\includegraphics[scale=0.5]{Screenshot 2023-12-11 191626.png}\\
\includegraphics[scale=0.5]{Screenshot 2023-12-11 191640.png}\\
\includegraphics[scale=0.5]{Screenshot 2023-12-11 191652.png}\\
\subsection{Analysis}
Using,
\begin{align*}
    2d\sin{\theta}=m\lambda \;\;\;\; where \;\;m&=1\\
    d&=0.282\;nm\\
    \theta&=incident\;angle\;(\circ)\\
    \lambda&=X\text{-}ray\;wavelength\;(nm)
\end{align*}
\begin{align*}
    K_{\beta}&=2\times0.282\times \sin{6.4}=0.062868478\;nm\\
    K_{\alpha}&=2\times0.282\times \sin{7.2}=0.070687944\;nm
\end{align*}
\\
Accepted values of $K_{\beta}$ and $K_{\alpha}$ for molybdenum are $0.0632872nm$ and  $0.0713590nm$ respectively.\\\\
Percentage error in $K$ values:\\
\begin{align*}
    \Delta K_{\beta}=\Bigg|\frac{0.0632872-0.062868478}{0.0632872}\Bigg|\times 100\%=\pm 0.662\%\\\\
    \Delta K_{\alpha}=\Bigg|\frac{0.071359-0.070687944}{0.071359}\Bigg|\times 100\%=\pm 0.0671\%
\end{align*}\\
Using $2d\sin{\theta}=m\lambda$ again to calculate $\lambda_{min}$:\\\\
For $35kV$: 
\begin{align*}
\lambda_{35kV}=2\times0.282\times\sin{3.4}=0.0334488nm
\end{align*}
\begin{align*}
\begin{tabularx}{0.9\textwidth} { 
  | >{\centering\arraybackslash}X 
  | >{\centering\arraybackslash}X
  | >{\centering\arraybackslash}X | }
 \hline
$V(V)$ & $\lambda_{min}\times 10^{-11} (m)$ & $1/\lambda_{min}\times10^{10}(m^{-1})$ \\
 \hline
 35 & 3.34488 & 2.98964\\
 30 & 3.93427 & 2.54177\\
 25 & 4.71943 & 2.11890\\
 20 & 5.89541 & 1.69624\\
\hline
\end{tabularx}
\\
\end{align*}
\newpage
\textbf{Graph of $1/\lambda_{min}$ against $V$:}\\\\
\includegraphics[scale=0.4]{desmos-graph.png}
\includegraphics[scale=0.4]{Screenshot 2023-12-11 214700.png}
\newpage
\subsection{Discussion}
The shapes of the graphs do agree with provided plot for the $30kV$ spectrum with both $K_{\beta}$ and $K_{\alpha}$ peaks aligning at $6.4^{\circ}$ and $7.2^{\circ}$ respectively. Furthermore the positions of 
the peaks $K_{\beta}$ and $K_{\alpha}$ do not seem to depend on x-ray tube voltage. Only the intensity of the spikes decrease as the voltage is decreased with $K_{\beta}$ disappearing for the $20kV$ spectra as expected since the higher energy $M\to K$ transition has a lower probability than the $L\to K$ transition. The percentage errors calculated in section 1.4 are within acceptable margins and can be decreased if precise angle measurements of more than 1 decimal place are taken. $\lambda_{min}$ seems to be inversely proportional to V as seen from the linear graph of $1/\lambda_{min}$ against $V$ thus agreeing with $eV=\frac{hc}{\lambda_{min}}$. Now $\frac{e}{hc}=\frac{1}{\lambda_{min} V}$. The Theoretical value for $\frac{e}{hc}$ is $8.065\times 10^5\; Cs^2m^{-3}kg^{-1}$. As seen from the Desmos table above, the gradient is 0.085. Multiplying by $10^7$ yields $8.5\times 10^5$.\\
Percentage error:
\begin{align*}
    \Delta=\Bigg|\frac{(8.065-8.5)\times 10^5}{8.065\times 10^5}\Bigg|\times 100\%=\pm 5.39\%
\end{align*}
Since the detector itself is moving through the given angle parameter, it is possible this does not happen smoothly resulting in irregular rotation rates which could explain the discrepancy in the $\frac{e}{hc}$ value obtained with the theoretical one as variable electron numbers strike the target. Also changing the current from $0.5mA$ to $1.0mA$ resulted in a higher intensity plot as expected since $Q=It$ but the positions of the K values and $\lambda_{min}$ remained constant proving the quantized nature of energy as it only takes discrete values with $h$ as the multiplying factor.
\subsection{Conclusion}
Overall this experiment successfully showed the discrete nature of energy through emitted X-rays further solidifying the wave-particle duality theory.  
\newpage
\section{X-ray absorption of Zr, Mo, Ag and In}
\subsection{Data}
\includegraphics[scale=0.7]{Screenshot 2023-12-11 230810.png}
\\
\subsection{Graphs}
\includegraphics[scale=0.5]{Screenshot 2023-12-11 231012.png}
\includegraphics[scale=0.5]{Screenshot 2023-12-11 231024.png}
\includegraphics[scale=0.5]{Screenshot 2023-12-11 231036.png}
\includegraphics[scale=0.5]{Screenshot 2023-12-11 231047.png}
\\
\subsection{Analysis}
\begin{align*}
    Using\;\;
    2d\sin{\theta}&=m\lambda\\
    \lambda_{Zr}&=2*0.282\times 10^{-9}\times \sin{7}\\
    &=6.873\times 10^{-11}\;m
\end{align*}
\begin{align*}
With,\;\;\;
    v_{K\;edge}&=\frac{c}{\lambda_{K\;edge}}\\
    v_{Zr}&=\frac{c}{6.873\times 10^{-11}}\\
    Now,\;\;\; \sqrt{v_{Zr}}&=\sqrt{\frac{c}{6.873\times 10^{-11}}}\\
    &=2.089\times 10^9 Hz^{1/2}
\end{align*}
\begin{align*}
\begin{tabularx}{1.2\textwidth} { 
  | >{\centering\arraybackslash}X 
  | >{\centering\arraybackslash}X 
  | >{\centering\arraybackslash}X
  | >{\centering\arraybackslash}X 
  | >{\centering\arraybackslash}X 
  | >{\centering\arraybackslash}X | }
 \hline
  Absorber & Z & Accepted values for K edge $\times 10^{-11}$ (m) & Obtained K edge values $\times 10^{-11}$ (m) & Crystal Angle ($\circ$) & $\sqrt{v_{K\;edge}}$ $\times 10^9$ $Hz^{1/2}$\\
 \hline
 Zr & 40 & 6.889 & 6.873 & 7.00 & 2.089\\
 Mo & 42 & 6.199 & 6.140 & 6.25 & 2.210\\
 Ag & 47 & 4.859 & 4.867 & 4.95 & 2.482\\
 In & 49 & 4.438 & 4.474 & 4.55 & 2.589\\
\hline
\end{tabularx}
\\
\end{align*}
Theoretical ratio/gradient for $\sqrt{v_{K\;edge}}$ and Z is given by $\sqrt{\frac{24cR}{25}}$ \\where $R=1.097\times10^7\;m^{-1}$\\
So,
\begin{equation*}
    \sqrt{\frac{24\times 2.998\times 10^8 \times 1.097\times10^7}{25}}=5.619\times 10^7\;Hz^{1/2}
\end{equation*}
\newpage
\subsection{Graph}
\includegraphics[scale=0.5]{desmos-graph 2.png}
\includegraphics[scale=0.5]{Screenshot 2023-12-12 012119.png}
\\\\
Actual Gradient of graph is $0.056\times10^9=5.6\times10^7$
\\
\begin{align*}
Now,\;\;\;\;\;\;\frac{\Delta\sqrt{v}}{\Delta(Z-b)}&=5.6\times10^7\\
    Z-b&=\frac{\sqrt{v}
    }{5.6\times10^7}\\
    b&=Z-\frac{\sqrt{v}}{5.6\times10^7}\\
    For\;\;\;\;\;
    b_{Zr}&=40-\frac{2.089\times10^9}{5.6\times10^7}\\
    &=2.697
    \\Then\;\;\;\;b_{Mo}=2.536,\;b_{Ag}&=2.679,\;b_{In}=2.768\\
    So\;\;\;\;b\approx 3.0
\end{align*}
\newpage
\section{Discussion}
Obtained K edge values are roughly correct to the nearest 10th place when compared to the accepted K edge values. Given that the midpoint angles of the best-fit lines for the gradual spikes were measured through eyeballing again, more precise data could be obtained through simple geometry to find the midpoint of the line thus reducing this error. The graph in section 2.4 also agrees with Mosley's law as $\sqrt{v_{K edge}}$ is proportional to (Z-b) and varies linearly to give a gradient value of $5.6\times10^7$. Comparing this to the theoretical ratio $\sqrt{v_{K\;edge}}=5.619\times 10^7$, the percentage error is $\Big|\frac{(5.619-5.6)\times10^7}{5.619\times10^7}\Big|\times100\%=\pm0.34\%$, which is well within the acceptable error range. Again this error can be decreased if a proper geometrical method was used for the midpoint angle. Smooth rotation of the detector in the apparatus as discussed in the 1st section also could have played a part in exaggerating this error. The b(shielding effect) value calculated is only correct to the nearest integer, which is enough to use in calculation with the Z(atomic number). However, it's value could be made more precise if the before-mentioned procedures were followed.
\section{Conclusion}
Overall the experiment was a success and it again proved the quantized nature of energy this time through the absorption of X-ray accross the 4 period 5 elements.

\end{document}
